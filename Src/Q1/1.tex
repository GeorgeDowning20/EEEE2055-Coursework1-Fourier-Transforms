% Derive the Fourier Transform of the S2(t) (see Figure 1 [Report: derivation of Fourier Transform including
% some comments to explain each step]
\begin{figure}[ht!]
	\centering
	% This file was created by matlab2tikz.
%
%The latest updates can be retrieved from
%  http://www.mathworks.com/matlabcentral/fileexchange/22022-matlab2tikz-matlab2tikz
%where you can also make suggestions and rate matlab2tikz.
%
\definecolor{mycolor1}{rgb}{0.00000,0.44706,0.74118}%
%
\begin{tikzpicture}

\begin{axis}[%
width=4.568in,
height=3.603in,
at={(0.766in,0.486in)},
scale only axis,
xmin=0,
xmax=0.003,
xlabel style={font=\color{white!15!black}},
xlabel={time[s]},
ymin=-0.1,
ymax=1.1,
ylabel style={font=\color{white!15!black}},
ylabel={[arb]},
axis background/.style={fill=white}
]
\addplot [color=mycolor1, line width=1.1pt, forget plot]
  table[row sep=crcr]{%
0	0\\
0.001	0\\
0.00199969996999705	0.999699969997\\
0.00200090009000897	0\\
0.003	0\\
};
\end{axis}
\end{tikzpicture}%
	\caption{$S_{2}(t)$}\label{fig:S2}
\end{figure}\FloatBarrier

\begin{figure}[ht!]
	\centering
	% This file was created by matlab2tikz.
%
%The latest updates can be retrieved from
%  http://www.mathworks.com/matlabcentral/fileexchange/22022-matlab2tikz-matlab2tikz
%where you can also make suggestions and rate matlab2tikz.
%
\definecolor{mycolor1}{rgb}{0.00000,0.44706,0.74118}%
%
\begin{tikzpicture}

\begin{axis}[%
width=4.568in,
height=3.603in,
at={(0.766in,0.486in)},
scale only axis,
xmin=0,
xmax=0.003,
xlabel style={font=\color{white!15!black}},
xlabel={time[s]},
ymin=-300,
ymax=1100,
ylabel style={font=\color{white!15!black}},
ylabel={[arb]},
axis background/.style={fill=white},
axis x line*=bottom,
axis y line*=left
]
\addplot [color=mycolor1, line width=1.1pt, forget plot]
  table[row sep=crcr]{%
0	0\\
0.000999699969952417	0\\
0.00100090009004816	1000\\
0.00199969997004246	1000\\
0.00200090009002452	0\\
0.00300000000004275	0\\
};
\addplot[ycomb, color=mycolor1, line width=1.1pt, mark=triangle, mark options={solid, rotate=180, mycolor1}, forget plot] table[row sep=crcr] {%
0.002	-200\\
};
\addplot[forget plot, color=white!15!black, line width=1.1pt] table[row sep=crcr] {%
0	0\\
0.003	0\\
};
\end{axis}
\end{tikzpicture}%
	\caption{$\frac{dS_{2}\left(t\right)}{dt}$}\label{fig:dS2}
\end{figure}\FloatBarrier



Figure \ref{fig:S2} shows the triangular pulse $S_{2}(t)$ and Figure \ref{fig:dS2} shows the derivative of $S_{2}(t)$. where the gradient of the triangle represents a rise of 1 in 1ms.

To calculate the fourier transform of the signal we can split the signal allowing us to differentiate the the different components to create delta functions. The first derivative of $S_{2}(t)$ is shown in \eqref{eq:first_derivative_of_S2}. This signal consists of a rectangular pulse with a width of 1ms and a height of 1000 can be differentiated again to form two unit impulses. Also there is unit impulse at 0.002ms.

The second derivative of $S_{2}(t)$ as a summation of delta functions is shown in \eqref{eq:second_derivative_of_S2}. 

\begin{equation}
	\begin{split}
		\frac{d^{2}S_{2}\left(t\right)}{dt^{2}} &= \frac{d}{dt}\left[\frac{dS_{2}\left(t\right)}{dt}\right]\\
		&= \left(\frac{1}{m}\delta\left(t - m\right) - \frac{1}{m}\delta\left(t - 2m\right)\right) + \frac{d}{dt}\left[\delta\left(t - 2m\right)\right]\\
		&= 1000\delta\left(t - 0.001\right) - 1000\delta\left(t - 0.002\right) + \frac{d}{dt}\left[\delta\left(t - 0.002\right)\right]
	\end{split}
	\label{eq:second_derivative_of_S2}
\end{equation}

In order to calculate the fourier transform of the Signal $S_{2}(t)$ we can use the derivative theorem. The derivative theorem states that the fourier transform of the derivative of a signal is the fourier transform of the signal divided by the frequency. Application of derivative theorem is shown in \eqref{eq:derivative_theorem}. Where the first line is the derivative theorem and the second line is the application of the derivative theorem to the second derivative of $S_{2}(t)$.

\begin{equation}
	\begin{split}
		\left(j\omega\right)^{2} \mathcal{F}\left\{S\left(t\right)\right\} =& \mathcal{F}\left\{\frac{d^{2}S_{2}\left(t\right)}{dt^{2}}\right\}\\
		\therefore \widetilde{S} \left\{\omega\right\} =& \frac{1}{\left(j\omega\right)^{2}} \cdot \mathcal{F}\left\{1000\delta\left(t - 0.001\right) - 1000\delta\left(t - 0.002\right)\right\}\\
		&+ \frac{1}{\left(j\omega\right)} \cdot \mathcal{F}\delta\left(t - 0.002\right)\\
	\end{split}
	\label{eq:derivative_theorem}
\end{equation}

The Fourier transform is evaluated using integration as shown in \eqref{eq:fourier_transform_of_S2}. 

\begin{equation}
	\begin{split}
		\widetilde{S} \left\{\omega\right\}
		=& \frac{1000}{\left(j\omega\right)^{2}} \cdot \int_{-\infty}^{\infty}\delta\left(t - 0.001\right)e^{-j\omega t}dt\\
		&  + \frac{-1000}{\left(j\omega\right)^{2}}\cdot  \int_{-\infty}^{\infty}\delta\left(t - 0.002\right)e^{-j\omega t}dt\\
		&  + \frac{-1}{j\omega} \cdot \int_{-\infty}^{\infty}\delta\left(t - 0.002\right)e^{-j\omega t}dt\\
	\end{split}
	\label{eq:fourier_transform_of_S2}
\end{equation}

The equation given in \eqref{eq:fourier_transform_of_S2} is then evaluated, using the properties of the delta function, to give the following equation \eqref{eq:fourier_transform_of_S2_evaluated}. 

\begin{equation}
	\begin{split}
		\widetilde{S} \left\{\omega\right\} =& \frac{-1000} {\omega^2} \cdot e^{-j\omega 0.001}\\
		& + \frac{1000} {\omega^2}  \cdot e^{-j\omega 0.002m}\\
		& + \frac{-1} {j\omega}  \cdot e^{-j\omega 0.002m}\\
		=& \frac{1}{\left(j \omega\right)^{2} m}\left[e^{-j\omega 0.001m} - e^{-j\omega 0.002m}\right] - \frac{1}{\left(j \omega\right)}\left[e^{-j\omega 0.001m} - e^{-j\omega 0.002m}\right]\\
		\label{eq:fourier_transform_of_S2_evaluated}
	\end{split}
\end{equation}

The equation given in \eqref{eq:fourier_transform_of_S2_evaluated} is then simplified to give the following equation \eqref{eq:fourier_transform_of_S2_simplified}.

\begin{equation}
	\widetilde{S} \left\{\omega\right\} = \frac{je^{-\frac{j\omega}{500}}}{\omega}+\frac{1000e^{-\frac{j\omega}{500}}-1000e^{-\frac{j\omega}{1000}}}{\omega^2}\label{eq:fourier_transform_of_S2_simplified}
\end{equation}

Converting the equation into polar form to avoid imaginary exponentials gives the following equation \eqref{eq:fourier_transform_of_S2_polar}.

\begin{equation}
		\widetilde{S} \left\{\omega\right\} = \frac{t\sin \left(\frac\omega{500}\right)-2000\sin \left(\frac\omega{2000}\right)\sin \left(\frac{3t}{2000}\right)}{t^2}+j\frac{t\cos \left(\frac\omega{500}\right)-2000\sin \left(\frac\omega{2000}\right)\cos \left(\frac\omega{2000}-\frac\omega{500}\right)}{t^2}
	\label{eq:fourier_transform_of_S2_polar}
\end{equation}

\begin{figure}[ht!]
	\centering
	% This file was created by matlab2tikz.
%
%The latest updates can be retrieved from
%  http://www.mathworks.com/matlabcentral/fileexchange/22022-matlab2tikz-matlab2tikz
%where you can also make suggestions and rate matlab2tikz.
%
\definecolor{mycolor1}{rgb}{0.00000,0.44700,0.74100}%
\definecolor{mycolor2}{rgb}{0.85000,0.32500,0.09800}%
%
\begin{tikzpicture}

\begin{axis}[%
width=4.568in,
height=3.603in,
at={(0.766in,0.486in)},
scale only axis,
xmin=0,
xmax=35000,
ymin=-55,
ymax=-10,
axis background/.style={fill=white},
axis x line*=bottom,
axis y line*=left
]
\addplot [color=mycolor1, forget plot]
  table[row sep=crcr]{%
159.200000000001	-12.8021258830777\\
201.022358800805	-12.9465751484349\\
242.005337891715	-13.1212883452026\\
283.72265444247	-13.3332358744083\\
323.930299018695	-13.5705042116933\\
364.000703126992	-13.8396507787256\\
404.714037925685	-14.147062347176\\
445.235423485981	-14.4875609173469\\
484.648136160638	-14.8523464937789\\
524.760432606057	-15.258174392533\\
565.188496173265	-15.7029843393466\\
605.512682058208	-16.1828391500931\\
645.283961998372	-16.691717398553\\
684.031649426361	-17.2214104244231\\
725.106038538317	-17.8188814910427\\
764.582836879039	-18.4267696371062\\
806.208862400894	-19.1010628747354\\
850.101125035555	-19.8445420997159\\
905.93746145433	-20.826204748395\\
1018.00255782626	-22.8078950220188\\
1050.90312391032	-23.3531271791981\\
1079.13106439784	-23.7920454560044\\
1108.11722570137	-24.2067208403678\\
1131.86574162414	-24.5141323932112\\
1156.12322175706	-24.7943701115873\\
1180.9005739214	-25.042909308464\\
1206.20893970845	-25.2562736236214\\
1232.0596994895	-25.4324635211851\\
1258.46447753327	-25.5712651895192\\
1285.43514723297	-25.6743720696504\\
1312.98383644536	-25.7452909079002\\
1348.25143509741	-25.7964546264957\\
1391.82523075963	-25.819152432854\\
1531.17964282111	-25.855918073612\\
1572.30812255426	-25.9110220710063\\
1614.54133996669	-26.0034128961306\\
1649.14323932431	-26.1088295425252\\
1684.4867062154	-26.245274455061\\
1720.58763347868	-26.4147882955658\\
1757.46225455889	-26.6186231586544\\
1795.12715080634	-26.8571046431935\\
1833.5992589331	-27.1294228887709\\
1882.85092598167	-27.5138384038728\\
1943.70230679642	-28.0279602308656\\
2060.41676720788	-29.029655251994\\
2104.57440619107	-29.3716873391713\\
2138.31258271326	-29.6055859071785\\
2172.59161184284	-29.8133071154807\\
2207.42016392223	-29.9899491389806\\
2242.80704828705	-30.1324208240985\\
2278.76121549431	-30.2399470258097\\
2315.29175958631	-30.3142988561412\\
2352.40792039084	-30.3596857844132\\
2402.82334184491	-30.3861247633577\\
2587.93923239034	-30.4331321377394\\
2643.40242224255	-30.5069816501455\\
2685.77848520257	-30.5952880669101\\
2728.83387367764	-30.7152447615772\\
2772.57947785254	-30.8683240304927\\
2817.02636249127	-31.054329630384\\
2862.18576973569	-31.2711791308975\\
2923.52644990015	-31.6006547605175\\
3099.07689860088	-32.587035544595\\
3148.75782371013	-32.8206777882842\\
3199.23517769175	-33.0157759529102\\
3233.33550402625	-33.1203451024194\\
3285.1687125217	-33.2366278060945\\
3337.8328528829	-33.3067799924465\\
3391.34124567759	-33.33943413421\\
3482.43487426978	-33.3483881137545\\
3575.97533689279	-33.365409842474\\
3633.30136290554	-33.408202845123\\
3691.54637547407	-33.489101279898\\
3750.72510676018	-33.6147375500805\\
3810.85252509512	-33.7871779578963\\
3871.9438387657	-34.0033853825698\\
3954.92505932094	-34.3439830350471\\
4082.74327772453	-34.8873767374971\\
4148.19323900575	-35.1269514506057\\
4192.40841397478	-35.2604170226805\\
4237.09487405153	-35.3679029029081\\
4282.25764261663	-35.4476810506749\\
4327.9017965944	-35.5006991844311\\
4397.28186429581	-35.5385398763174\\
4515.39571127947	-35.5462266619143\\
4587.78147113976	-35.5609671072998\\
4661.32763831884	-35.6127032875484\\
4711.01226129595	-35.6761325545704\\
4761.2264676776	-35.7659619763581\\
4811.97590224009	-35.8826631725024\\
4889.11610309625	-36.1036150215841\\
4993.89676071046	-36.4564491116143\\
5100.92301568391	-36.8136279556093\\
5155.29324162258	-36.9685639081363\\
5210.24299433699	-37.0959063311784\\
5265.77845094484	-37.1905257469225\\
5321.90585440505	-37.2517561891291\\
5378.63151421967	-37.2837282967957\\
5464.85570232191	-37.2957821405471\\
5581.97527203176	-37.3053491157407\\
5641.4729856399	-37.3329716173976\\
5701.60487939971	-37.3875169068597\\
5762.37771297372	-37.4738437908418\\
5823.79831807498	-37.5929958747729\\
5885.87359923516	-37.7418911385503\\
5980.22932355363	-38.0040443158359\\
6108.39407726217	-38.3608808986974\\
6173.50283602641	-38.5102509107928\\
6239.30558250891	-38.6247230126937\\
6305.809713855	-38.6996564427427\\
6373.0227060555	-38.7380716636071\\
6475.18785849882	-38.7509986378936\\
6613.96027209715	-38.7680583677175\\
6684.45781013831	-38.8107910451399\\
6755.70677435463	-38.8896895049511\\
6827.71517412222	-39.007405325021\\
6900.4911041883	-39.1597745198596\\
7161.37006348581	-39.7643146558257\\
7237.70238749558	-39.8833103552315\\
7314.84833007792	-39.9570237256667\\
7392.81656351148	-39.989800712854\\
7672.30830725065	-40.0386943683006\\
7754.08667066717	-40.1131051282209\\
7836.73670144314	-40.2321722479646\\
7920.26769059338	-40.3899118020272\\
8133.01122511005	-40.8309328401738\\
8219.70017464159	-40.9630549695175\\
8307.31313297663	-41.0450268992136\\
8395.85994903215	-41.0797234221645\\
8667.20361409661	-41.1189364314814\\
8759.58646663251	-41.1928411667977\\
8852.95401871194	-41.3170563204803\\
8947.31676614844	-41.4821820849938\\
9139.07039090813	-41.8337837117797\\
9236.48282401	-41.9600076856877\\
9334.93356644935	-42.0297714314147\\
9434.43368546108	-42.051614234002\\
9636.62691322149	-42.0696866113394\\
9739.3427513056	-42.1296567671889\\
9843.15342718805	-42.2455303381394\\
9948.07061063497	-42.4087447431593\\
10107.5469215363	-42.6776307384534\\
10215.2822486761	-42.8177006984879\\
10324.1659158438	-42.897228314323\\
10434.2101630804	-42.9218406397995\\
10601.4797180631	-42.9295543192093\\
10714.4798252561	-42.9743148574162\\
10828.6843892386	-43.079148596622\\
10944.1062481945	-43.2386754188228\\
11119.5498853465	-43.5058627911567\\
11238.0720504024	-43.6371855288271\\
11357.8575313079	-43.7012899971778\\
11478.9197936195	-43.7137880106711\\
11601.2724464221	-43.7196132451609\\
11724.9292438583	-43.7665480803771\\
11849.9040866747	-43.8795302806939\\
11976.2110237845	-44.0475099370997\\
12103.864253847	-44.2268414557329\\
12232.8781268635	-44.3635454823889\\
12297.8998306342	-44.4048944835631\\
12428.9819093691	-44.4372165817731\\
12628.2294077782	-44.4486717318941\\
12695.3525314769	-44.4712673190552\\
12762.832436288	-44.511581227518\\
12898.8701849552	-44.6463835765353\\
13175.3111759155	-44.9887235311362\\
13245.3422161566	-45.0462912254152\\
13315.745494027	-45.0834293127664\\
13457.6766874259	-45.1071941170921\\
13601.1207111555	-45.1121276412159\\
13673.4150667976	-45.1289560112491\\
13746.0936903224	-45.1637822105949\\
13819.1586242379	-45.2188580693401\\
13966.4556476143	-45.3793637035742\\
14115.322695164	-45.5578081636522\\
14190.3502006614	-45.6305945606073\\
14265.7765016169	-45.6832005651813\\
14341.6037177575	-45.7139978149025\\
14494.469430896	-45.72844545505\\
14648.9645243126	-45.7413430394663\\
14726.8285087297	-45.7701989062225\\
14805.1063654079	-45.8205085968002\\
14962.9125067	-45.9773496169291\\
15122.4006877975	-46.155126199028\\
15202.7811385488	-46.225618684035\\
15283.5888373936	-46.2739638328094\\
15364.8260552938	-46.2994485533491\\
15694.1159605145	-46.3331627452208\\
15777.5352628524	-46.3748925227883\\
15861.3979657628	-46.4395152964207\\
16201.3301077876	-46.7725346417246\\
16287.4454173684	-46.8206650250504\\
16374.0184576723	-46.8446542012352\\
16724.9367802492	-46.8866106222486\\
16813.8352286463	-46.9375966048174\\
16993.0522090763	-47.10014465516\\
17174.179445295	-47.2688167888045\\
17265.465763757	-47.3241419789447\\
17357.2372985268	-47.3533800891091\\
17729.2273883825	-47.3974184978579\\
17823.4639661834	-47.4502080883503\\
18013.4424778902	-47.6166786892754\\
18109.1897508165	-47.7052564701589\\
18205.4459514664	-47.7771679905454\\
18302.2137849552	-47.8228165273467\\
18497.2952428808	-47.8453277248736\\
18694.4560544667	-47.8660325249293\\
18793.8231348154	-47.9097038730069\\
18893.7183833347	-47.9799537668514\\
19196.6012864539	-48.2325124397357\\
19298.6374311214	-48.2800785225299\\
19401.2159308996	-48.2998861753549\\
19608.0115422829	-48.3062468359713\\
19712.2344655241	-48.3276864423315\\
19817.0113673115	-48.3769666092667\\
20028.2389004769	-48.5433397142442\\
20134.6954680426	-48.6295046749765\\
20241.7178866965	-48.6937583952022\\
20349.3091641202	-48.7277962838089\\
20675.5264661382	-48.7509473651153\\
20785.4235764701	-48.7905917553435\\
20895.9048254877	-48.859848494998\\
21118.6321756283	-49.0366993089374\\
21230.8845361215	-49.1044670366864\\
21343.7335542169	-49.1409353840354\\
21685.8923527732	-49.1663325142363\\
21801.159884579	-49.2094962269803\\
22033.5362574442	-49.3731236913409\\
22150.6516290432	-49.4577350216823\\
22268.3895067212	-49.5163756076763\\
22386.7531992995	-49.5421463776402\\
22625.3713524723	-49.5505464508969\\
22745.6325190207	-49.5779987489113\\
22866.5329125659	-49.6382441802925\\
23233.1035225597	-49.8817873031112\\
23356.5949821532	-49.9164295665396\\
23731.0217176238	-49.9494672888504\\
23857.1597734678	-50.0063845340374\\
24239.6110022709	-50.2479806390984\\
24368.4523746577	-50.2802489603819\\
24628.193260241	-50.2901749860866\\
24759.1000730096	-50.3195787309487\\
24890.7026978277	-50.3844422276889\\
25156.0101971356	-50.5578841677561\\
25289.7225276376	-50.6130528140129\\
25424.1455824233	-50.6322037249774\\
25695.1389958435	-50.6478016719338\\
25831.7169702831	-50.6960811062818\\
26245.8220855268	-50.9344909707943\\
26385.3271186976	-50.9639612444807\\
26666.5656695438	-50.9761971463959\\
26808.3070909464	-51.0171909195378\\
27238.0678037228	-51.2546586414792\\
27382.8469438115	-51.285620630526\\
27674.7179582537	-51.2988665365774\\
27821.8180351413	-51.3427545520026\\
28267.826218685	-51.5753739404645\\
28418.078857808	-51.5982511104994\\
28720.9843052125	-51.6177922158895\\
28873.6456261322	-51.6749826183441\\
29181.4069139499	-51.8448775387369\\
29336.5155299511	-51.8902687839618\\
29649.2105015694	-51.9046386209702\\
29806.8056449424	-51.9428156716131\\
30284.6349236096	-52.1701696542368\\
30770.1242253691	-52.2184977432844\\
30933.6773877255	-52.2890603590058\\
31098.0998880384	-52.3790873768558\\
31263.3963471203	-52.4456486390081\\
31429.5714103447	-52.4681517580175\\
31596.6297477767	-52.4702793197175\\
31764.5760543046	-52.4960841339052\\
};
\addplot [color=mycolor2, only marks, mark=o, mark options={solid, mycolor2}, forget plot]
  table[row sep=crcr]{%
0	-15.563025530344\\
333.333333333332	-13.6238715620311\\
666.666666666668	-16.9750405995255\\
1000	-22.4957227642408\\
1333.33333333333	-25.7724527512401\\
1666.66666666667	-26.1671837232607\\
2000	-28.5163223019699\\
2333.33333333333	-30.3330722950668\\
2666.66666666667	-30.5460102123834\\
3000	-32.0381477860356\\
3333.33333333333	-33.2959208414686\\
3666.66666666667	-33.4435636294438\\
4000	-34.5369221376532\\
4333.33333333333	-35.4988907622574\\
4666.66666666667	-35.6123226309064\\
5000	-36.4751228555433\\
5333.33333333333	-37.253963003197\\
5666.66666666667	-37.3461943500442\\
6000	-38.0587475020875\\
6333.33333333333	-38.7130442570415\\
6666.66666666667	-38.7908069599835\\
7000	-39.3976833690285\\
7333.33333333333	-39.9617809163756\\
7666.66666666667	-40.0290237282788\\
8000	-40.55752222499\\
8333.33333333333	-41.0532790631041\\
8666.66666666667	-41.1125218133566\\
9000	-41.5805727187617\\
9333.33333333333	-42.0227595613651\\
9666.66666666667	-42.0757103113938\\
10000	-42.4957225008875\\
10333.3333333333	-42.8947882623143\\
10666.6666666667	-42.9426592883647\\
11000	-43.3235762029399\\
11333.3333333333	-43.6871841807879\\
11666.6666666667	-43.7308673094885\\
12000	-44.0793474235106\\
12333.3333333333	-44.413284603972\\
12666.6666666667	-44.4534552305631\\
13000	-44.7745895251865\\
13333.3333333333	-45.0833331831236\\
13666.6666666667	-45.1205152744296\\
14000	-45.4182832490187\\
14333.3333333333	-45.7053680263416\\
14666.6666666667	-45.7399761086454\\
15000	-46.0175477063749\\
15333.3333333333	-46.2858135498427\\
15666.6666666667	-46.3181814524687\\
16000	-46.578122129933\\
16333.3333333333	-46.8298843752382\\
16666.6666666667	-46.8602849614654\\
17000	-47.1047009370632\\
17333.3333333333	-47.3418725894262\\
17666.6666666667	-47.3705315117659\\
18000	-47.6011726080469\\
18333.3333333333	-47.8253521759616\\
18666.6666666667	-47.8524584152619\\
19000	-48.0707945289359\\
19333.3333333333	-48.283331541752\\
19666.6666666667	-48.3090448423791\\
20000	-48.5163224325152\\
20333.3333333333	-48.718366482055\\
20666.6666666667	-48.7428231252816\\
21000	-48.9401083966841\\
21333.3333333333	-49.1326468252737\\
21666.6666666667	-49.1559640423984\\
22000	-49.3441761448339\\
22333.3333333333	-49.5280632108334\\
22666.6666666667	-49.5503424832496\\
23000	-49.7302792240298\\
23333.3333333333	-49.9062589383284\\
23666.6666666667	-49.9275888349366\\
24000	-50.0999473814081\\
24333.3333333333	-50.2686718272489\\
24666.6666666667	-50.2891299077528\\
25000	-50.4545226343398\\
25333.3333333333	-50.6165663556421\\
25666.6666666667	-50.636221287783\\
26000	-50.7951894833823\\
26333.3333333333	-50.9510613276216\\
26666.6666666667	-50.9699736977527\\
27000	-51.1229977725525\\
27333.3333333333	-51.2731506889868\\
27666.6666666667	-51.2913746525373\\
28000	-51.4388831370961\\
28333.3333333333	-51.5837219146415\\
28666.6666666667	-51.6013058098542\\
29000	-51.7436824554934\\
29333.3333333333	-51.8835703980003\\
29666.6666666667	-51.9005577006174\\
30000	-52.0381475887189\\
30333.3333333333	-52.1734119706598\\
30666.6666666667	-52.1898418564379\\
31000	-52.3229563855639\\
31333.3333333333	-52.4538930475173\\
31666.6666666667	-52.4698009225795\\
};
\end{axis}
\end{tikzpicture}%
	\caption{$S_2\left(f\right)$}\label{fig:s2f} 
\end{figure}\FloatBarrier 

