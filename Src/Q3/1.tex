% Using convolution, compute an expression for the time-domain output waveform of the filter when Vi(t) = S2(t).
% This is the final question for the coursework and is more difficult problem to solve. The process you need
% to follow is exactly the same as in the lecture slides.
% What are the limits of the convolution integral (i.e. where is the product of F (τ ) and S2(t − τ ) non zero)?
% You will need to consider two different sets of limits for different ranges of t. You can start by sketching F (τ )
% S2(t − τ ) on a plot with τ on the horizontal axis to understand these limits.
% As a hint, the only real difference to the lecture example, is that you have a triangle pulse rather than a
% step input. Other than the obvious difference in function shape / different equation for Vin, the limits will be
% different (a step goes on to infinity, and also this pulse doesn’t start at t = 0 like the step in the lecture example).
% [Report: Derivation of time-domain output waveform. Even if you can’t quite get the correct answer for the
% final part, there are marks for attempting this.